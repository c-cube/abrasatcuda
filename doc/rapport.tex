\documentclass{article}

\usepackage{listings}
\usepackage[french]{babel}
\usepackage[pdftex]{color}
\usepackage[utf8]{inputenc}
\usepackage[T1]{fontenc}
\usepackage{fancybox}
\usepackage{verbatim}
\usepackage{lmodern}
\usepackage{hyperref}   % pour les urls


\lstset{
language=C,
keywordstyle=\bfseries\color[rgb]{0,0,1},
identifierstyle=\ttfamily,
commentstyle=\ttfamily\color[rgb]{0.133,0.545,0.133},
stringstyle=\ttfamily\color[rgb]{0.627,0.126,0.941},
showstringspaces=false,
basicstyle=\footnotesize,
numberstyle=\bfseries\footnotesize,
numbers=left,
stepnumber=0,
%numbersep=10pt,
tabsize=4,
breaklines=true,
breakatwhitespace=false,
%aboveskip={1.5\baselineskip},
columns=fixed,
extendedchars=true,
frame=single
}

% quelques commandes pratiques...
\newcommand{\commande}[1]{\fbox{\texttt{\$ #1}}}			% boite
\newcommand{\nbcommande}[1]{\texttt{\$ #1}}					% pas de boite
\newcommand{\multicommande}[1]{
	\framebox[\width]{
	\parbox{\textwidth}{#1}}}
\newcommand{\alert}[1]{\fbox{\textbf{#1}}}

\newcommand{\keyword}[1]{\textbf{\textcolor{blue}{#1}}}
\newcommand{\code}[1]{\texttt{\textcolor{green}{#1}}}
\newcommand{\cuda}{\textsc{CUDA}}


%-----------------------------------------------------
%----------------------------------------------------
%---------------------------------------------------



\author{Vincent Barrielle et Simon Cruanes}
\title{Rapport de projet : \textsc{SAT}-solver en \cuda}


\begin{document}

\maketitle
\tableofcontents%[pausesections] 
\newpage

\section{Introduction}
Ce projet vise à implémenter de manière efficace un \textsc{SAT}-solver de manière parallèle, plus particulièrement à l'aide de l'\textsc{API} \cuda de NVidia. Il est basé sur une méthode complète, c'est-à-dire parcourant exhaustivement l'arbre des possibilités d'affectations des variables, mais guidé par des heuristiques pour plus d'efficacité.

\section{Algorithme et implémentation efficace}
% description de l'algo et de son implémentation
% {{{

%}}}


\section{Implémentations parallèles}
% les différents parallélismes explorés et leur implémentation
%{{{

%}}}

\section{Performances comparées}
% benchmarking.
%{{{

%}}}


\section{Répartition du travail}
% rôle de chacun
%{{{
%}}}


\section{Conclusion}

\end{document}


